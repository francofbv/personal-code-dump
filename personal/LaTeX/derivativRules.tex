\documentclass{article}

\usepackage[english]{babel}
\usepackage[utf8]{inputenc}
\usepackage{amsmath,amssymb}
\usepackage{parskip}
\usepackage{graphicx}

% Margins
\usepackage[top=2.5cm, left=3cm, right=3cm, bottom=4.0cm]{geometry}
% Colour table cells
\usepackage[table]{xcolor}

% Get larger line spacing in table
\newcommand{\tablespace}{\\[1.25mm]}
\newcommand\Tstrut{\rule{0pt}{2.6ex}}         % = `top' strut
\newcommand\tstrut{\rule{0pt}{2.0ex}}         % = `top' strut
\newcommand\Bstrut{\rule[-0.9ex]{0pt}{0pt}}   % = `bottom' strut

%%%%%%%%%%%%%%%%%
%     Title     %
%%%%%%%%%%%%%%%%%
\title{Differentiation Rules}
\author{Franco Vidal \\ Math 124}
\date{\today}

\begin{document}
\maketitle
\maketitle
\section*{Definition of the derivative}
\begin{align}
    \label{eq:example_equation} % Equation label; can be used for referencing
    f'(x) = \lim_{h \to 0},. \frac{f(x + h) - f(x)}{h}
\end{align}
We can use this definition to compute an derivative without using any other derivative rules 

\begin{align}
    \label{eq:example_equation} % Equation label; can be used for referencing
    f(x) = 2x^2 - 16x + 35
\end{align}
\begin{align}
    \Label 
    f'(x) = \lim_{h \to 0} \frac{2(x+h)^2 - 16(x+h) + 35 - (2x^2 - 16x + 35)}{h}
\end{align}
\begin{align}
    \label 
    
\end{align}

\end{document}
